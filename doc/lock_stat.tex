\section{/proc/lock\_stat}

The kernel lock usage statistic is available via /proc/lock\_stat 
and has to be enabled manually on the most systems, by recompiling the kernel.
The reason for this, that this feature make use of Lockdep, which will
hook several kernel lock methods, maps the lock instances to a lock classes of the same type
and though create slightly additional load.
A lock class is a collection of actual lock instances, that are used in the same context,
for example in a file system driver, each inode has its own lock instance,
but they are classified as the same lock class as they are used in the same code regions.

According to the documentary the actual hooking mechanism is implemented as follows:

%\begin{figure}[H]
\begin{center}
    \begin{tikzpicture}[-,>=stealth',shorten >=1pt,auto,node distance=2.0cm,
        thin,main node/.style={line width=0,scale=0.5,font=\sffamily\Large\bfseries}]
            \node[main node] (0) {\_\_acquire() //hook};
            \node[main node] (1) [below of=0] {lock()};
            \node[main node] (2) [below of=1, xshift=4cm] {\_\_contended() //hook};
            \node[main node] (3) [below of=2] {$<$wait$>$};
            \node[main node] (4) [below of=3, xshift=-4cm] {\_\_acquired() //hook};
            \node[main node] (5) [below of=4] {$<$hold$>$};
            \node[main node] (6) [below of=5] {\_\_release() //hook};
            \node[main node] (7) [below of=6] {unlock()};

            \path[->, thick,every node/.style={font=\sffamily\small}]
                (0) edge node {} (1)
                (1) edge node {} (2)
                (2) edge node {} (3)
                (3) edge node {} (4)
                (1) edge node {} (4)
                (6) edge node {} (7);
            \path[dashed,->, thick,every node/.style={font=\sffamily\small}]
                (4) edge node {} (5)
                (5) edge node {} (6);

    \end{tikzpicture}
\end{center}
%\end{figure}

The hooks are used to track the state of lock manipulating methods.
A lock is called contended, if an acquisition has to wait.
Wait time is defined as the the time between the \_\_contended and \_\_acquired hooks.
As we are dealing with multiple lock instances, wait and hold times of more
than one second per second is possible.
The line of code, that tried to acquire a lock, is contended called contention point.
In addition the acquisition, that caused the contention is also tracked.
Lockstat will store up to four contention points and points, that caused a contention.


\subsection{Wrstat Module}
%wait and holdtime in $\my$s 
%time unit converted to ms
%provided information
%DFA
    \subsubsection{Source Files}
    The lock\_stat module contains the following files:
    \begin{description}
        \item[lock\_stat-reset.sh test\_directory]
            This script will be called by the presampling method and will reset lock\_stat.
            This done by writing "0" to /proc/lock\_stat.
            According to the documentary this should also disable lock\_stat, but this 
            behavior was not observed on any system we have used.

        \item[lock\_stat.py]
            The actual modules used by wrstat.
    \end{description}

    \subsubsection{Filtering}
    This module supports filtering for symbol names, but it will not use
    the actual lock class name, but corresponding contention points. So if a lock class
    was contended or caused a contention by a method listed in the filter, it will pass.
    Otherwise it will be discarded.
    Filters can be passed by the lock\_stat\_filter option in wrstat.config as space separated list.
    For each filter separated graphs are created.

    \subsubsection{Created Files in Test Directory}
    For the graphs ( .svg files) the corresponding patterns are listed.
    This module will create the following files in the test directory.
    \begin{description}
        \item[samples/lock\_stat\_*.svg:]
            Snapshot of /proc/lock\_stat took at the specified time step.
        \item[lock\_stat-*-holdtime.svg and lock\_stat-*-waittime.svg]
            Stacked bar chart of the locks either for total wait- or holdtime.
        \item[lock\_stat-*-holdtime-top.svg and lock\_stat-*-waittime-top.svg:]
            Time series graph for either wait- or holdtime for the top waiting locks.
        \item[lock\_stat-*-holdtime-top-*.svg:]
            For each lock class listed in the top-graphs, a corresponding detailed lock view is created.
            This shows, wait- and holdtime, acquisition and contentions of the lock as time series.
            In addition there are bar charts for contentions points and points, we are contendning with.
        \item[lock\_stat-filter-*.svg:]
            Filtered graphs for the given filter, or all filters.
    \end{description}
