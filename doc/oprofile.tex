\section{OProfile}
    %TODO how to use short
    OProfile is a system wide performance analysis tool, to analyze the runtime
    of single applications and even of kernel methods.
    It will periodically issue non maskable interrupts (NMI)
    and update a counter corresponding to the currently interrupted code.

\subsection{Wrstat Module}
    \subsubsection{Source Files}
    The OProfile module contains the following files:
    \begin{description}
        \item[oprofile-init.sh test\_directory]
            This script will initialize OProfile for the specified test directory
            using either opcontrol or operf.
            This script will be called by oprofile.py presampling().

        \item[oprofile-deinit.sh test\_directory]
            This script will deinitialize OProfile and write the opreport
            output to samples/oprofile.

        \item[oprofile.py]
            The actual modules used by this tool.

    \end{description}

    \subsubsection{Created Files in Test Directory}
    The created graphs will show the CPU usage by applications ( actual binaries)
    or symbol name ( methods)

    For the graphs ( .svg files) the corresponding patterns are listed.
    This module will create the following files in the test directory.
    \begin{description}
        \item[oprofile\_data/]
            Contains all collected data and will be passed to OProfile
            as --session-dir argument. It can be used to create custom
            results with opreport.

        \item[samples/oprofile]
            This is the output of opreport -l ( list all symbols) and
            the actual file that is beeing parsed.

        \item[oprofile-*-app.svg and oprogile-*-sym.svg]
            Graphs grouped either by symbol or application names.

        \item[oprofile-*-aggregated-*.svg and oprofile-*-cpu*-*.svg]
            Are either aggregated or per cpu graphs.

        \item[oprofile-filter-all-*.svg and oprofile-filter-path\_to\_filter-*.svg]
            Graphs filtered by function names for a particuar or all filter in 
            conjunction.
            If a filter creates an empty data set, its graph will not be created.

        \item[oprofile-app-filter-all*.svg and oprofile-app-filter-appname-*.svg]
            Graphs filtered by application names for a particuar or all filter in 
            conjunction.
            If a filter creates an empty data set, its graph will not be created.
    \end{description}

    \subsubsection{Configuration}
    There are are the following configuration options available in wrstat.config.
    \begin{description}
        \item[oprofile\_use\_operf [true,false]]
            If value is true, operf will be used, opcontrol otherwise.
            Although opcontrol is deprecated, it seems to work more stable than operf.
        \item[oprofile\_event]
            The default event, that is used by OPRofile for triggering NMI.
            The default event is 
            \begin{lstlisting}
CPU_CLK_UNHALTED:100000:0:1:1
            \end{lstlisting}
            which will trigger a NMI every 100.000 cycles, the CPU is running
            in kernel or user mode.
            This can be used to adopt the sampling rate used by OPRofile.
        \item[oprofile\_vmlinux]
            Path to the current vmlinux ELF binary.
            It is not possible to use the vmlinuz file as this is zipped and also
            contains a raw decompressor, that will uncompress the kernel at runtime.
            Since verion 1.0.0 OProfile does not require the vmlinux image anymore 
            to resolve symbol names, as it will use /proc/kallsyms instead.
            %TODO where to find?
        \item[oprofile\_missing\_binaries]
            Comma separated list of paths to search for additional binaries ( e.g. .ko files).
            Since Kernel version 2.6 this is needed to find modules, which
            are usually located in /lib/modules/version/kernel.
            It is also possible to pass user space binaries.
        \item[oprofile\_sym\_filter]
            Space separated list of symbol name filters invoked by this module.
            Those are files containing function names to show
            and are specified by thier path relative to the tool directory.
        \item[oprofile\_app\_filter]
            Space separated list of applications to filter for.
    \end{description}


    \subsubsection{Filtering}
    As OProfile is configured to run in a system wide configuration there might be alot
    of methods listed, that are not of interest and mess up the results.
    For this reason this module implements filtering for symbol and application names.
    The most basic filtering method is the application filter,
    that will create individual plots for a given set of application names provided by OPRofile.
    Filtering for symbol names allows to create extra graphs for individual subsets of functions.
    A single filter is a file containing all function names of interest
    and can be created from a list of non stripped binaries by wrstat-filer-create.
    This even allows to monitor small subsystems of an application by using 
    object files of specific source file.

\subsection{Encountered Problems}
    %TODO output description
    \subsubsection{opcontrol vs. operf}
    There are basically two different ways to start the OProfile daemon: operf and opcontrol,
    which is deprecated since version 0.9.8 and was removed in 1.0.0.
    Both offer mostly the same functionality but are slightly different in the usage:
    opcontrol is a tool to configure, start and stop the daemon
    whereas operf will is a one-line call to start the daemon, with specified options.
    Since either OProfile or operf could cause problems or is even unavailable
    ( depending on operation system, machine and version), wrstat supports both.
    To choose one option just modify oprofile\_use\_operf in wrstat.config.

    \subsection{Timer Mode}
    Hardware performance monitor counters ( PMC) will count hardware events
    and issues an interrupt on counter overflow.
    Per default OPRofile will use events such as CPU\_CLK\_UNHALTED to periodically
    interrupt the CPU.
    On some machines and CPU types ( e.g. on virtual machines) PMC may be not available,
    so usual timer interrupts can be used, by passing the timer=1 to the OPRofile kernel module (prior v1.0.0).

%TODO call graph
